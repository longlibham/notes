% this is a Latex manuscript

%=======================================================================
%   Copyright (C) 2025 Univ. of Bham  All rights reserved.
%   
%   		FileName:		notes_on_particle_physics.tex
%   	 	Author:		LongLI <long.l@cern.ch>
%   		Time:			2025.10.23
%   		Description:
%
%======================================================================

%defination 
\documentclass[a4paper,12pt]{article}


%package
\usepackage{geometry}
\usepackage{graphicx}
\usepackage{fullpage}
\usepackage{siunitx}
\usepackage{amsmath}
\usepackage{verbatim}
\usepackage{overpic}
\usepackage{url}
\usepackage{xspace}
\usepackage{subfigure}
\usepackage{caption}
\usepackage{hyperref}
\renewcommand{\thefootnote}{\fnsymbol{footnote}}
\graphicspath{{fig/}}

\sisetup{
	free-standing-units=true,
	space-before-unit=true,
	use-xspace=true
}

% hype link setting
\hypersetup{colorlinks,breaklinks,
  linkcolor=black,citecolor=blue,
  filecolor=blue,urlcolor=blue,
  pdfpagemode=UseNone
}

%fot list
\usepackage{listings}
\usepackage{color, xcolor}
\usepackage{listings}
%\lstset{emph={define}, emphstyle=\color{yellow}}
\lstset{backgroundcolor=\color{white}}
\lstset{language=Python, 
    breaklines=true,
    basicstyle=\ttfamily\footnotesize, 
    keywordstyle=\color{keywords},
    commentstyle=\color{comments},
    stringstyle=\color{red},
    showstringspaces=false,
    identifierstyle=\color{blue},
    %procnamekeys={def,class}
    morekeywords={*,...,-},
    numbers=left,
    xleftmargin=2em,
    frame=single,
    framexleftmargin=1.5em
}



% personal macros
\newcommand{\eq}[1]{\eqref{eq:#1}}
\newcommand{\tab}[1]{Table~\ref{tab:#1}}
\newcommand{\fig}[1]{Figure~\ref{fig:#1}}
\newcommand{\sect}[1]{Section~\ref{sect:#1}}


%title
\title{Long's notes and thoughts on Particle Physics}
\author{Long LI\footnote{long.l@cern.ch}}
%\emailAdd{long.l@cern.ch}
\date{Oct. 2025}

\begin{document}


%Add content for page two here (useful for two-sided printing)
\thispagestyle{empty}
\maketitle

\tableofcontents
\newpage

\section{Book: An Introduction To The Standard Model Of Particle Physics}
\label{sect:book1}

\subsection{The construction of the Standard Model}
\begin{enumerate}
    \item Bosons
        \begin{itemize}
            \item Scalar \\
            $J^{P} = 0^{+}$, spin 0, parity +1. Example: Higgs Boson
            \item Pseudo-Scalar \\
            $J^P = 0^-$, example Pions, Kaons, $\eta , \eta^{'}$
            \item Vector \\
            $J^P = 1^{-}$, example: $\gamma$, W, Z, $\rho$, $\Omega$, $J/\Psi$, $\Upsilon$.
        \end{itemize}
    \item Deep In-elastic Scattering (DIS)
        \begin{itemize}
            \item $Q^2$ \\
            Negative square of the four-momentum transferred from the incoming electron to the target proton via the virtual photon.\\
            \begin{equation}
                Q^2 = -q^2
            \end{equation}
            q, four-momentum of the virtual photon ($q = p_{electron-in} - p_{electron-out}$)  
            \item Bjorken Scaling variable \\
            \begin{equation}
                x = \frac{Q^2}{2pq}
            \end{equation}
            Physically represents the fraction of the proton's momentum carried by the parton that the photon hit. $p$, the four-momentum of the target proton.
        \end{itemize}
        
\end{enumerate}



\end{document}
