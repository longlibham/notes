% this is a Latex manuscript

%=======================================================================
%   Copyright (C) 2025 Univ. of Bham  All rights reserved.
%   
%   		FileName:		notes_on_particle_physics.tex
%   	 	Author:		LongLI <long.l@cern.ch>
%   		Time:			2025.10.23
%   		Description:
%
%======================================================================

%defination 
\documentclass[a4paper,12pt]{article}


%package
\usepackage{geometry}
\usepackage{graphicx}
\usepackage{fullpage}
\usepackage{siunitx}
\usepackage{amsmath}
\usepackage{verbatim}
\usepackage{overpic}
\usepackage{url}
\usepackage{xspace}
\usepackage{subfigure}
\usepackage{caption}
\usepackage{hyperref}


\renewcommand{\thefootnote}{\fnsymbol{footnote}}
\graphicspath{{fig/}}

\sisetup{
	free-standing-units=true,
	space-before-unit=true,
	use-xspace=true
}

% hype link setting
\hypersetup{colorlinks,breaklinks,
  linkcolor=black,citecolor=blue,
  filecolor=blue,urlcolor=blue,
  pdfpagemode=UseNone
}

%fot list
\usepackage{listings}
\usepackage{color, xcolor}
\usepackage{listings}
%\lstset{emph={define}, emphstyle=\color{yellow}}
\lstset{backgroundcolor=\color{white}}
\lstset{language=Python, 
    breaklines=true,
    basicstyle=\ttfamily\footnotesize, 
    keywordstyle=\color{keywords},
    commentstyle=\color{comments},
    stringstyle=\color{red},
    showstringspaces=false,
    identifierstyle=\color{blue},
    %procnamekeys={def,class}
    morekeywords={*,...,-},
    numbers=left,
    xleftmargin=2em,
    frame=single,
    framexleftmargin=1.5em
}



% personal macros
\newcommand{\eq}[1]{\eqref{eq:#1}}
\newcommand{\tab}[1]{Table~\ref{tab:#1}}
\newcommand{\fig}[1]{Figure~\ref{fig:#1}}
\newcommand{\sect}[1]{Section~\ref{sect:#1}}


%title
\title{Long's notes and thoughts on Pixel Detector}
\author{Long LI\footnote{long.l@cern.ch}}
%\emailAdd{long.l@cern.ch}
\date{Oct. 2025}

\begin{document}


%Add content for page two here (useful for two-sided printing)
\thispagestyle{empty}
\maketitle

\tableofcontents
\newpage

\section{Book: Primary Analogue circuits.}
\label{sect:book1}

\subsection{Transistors}
\subsubsection{BJT}

\begin{figure}[htb!]
  \centering
  \includegraphics[width=0.6\textwidth]{plots/BJT_input_output}
\end{figure}

\begin{enumerate}
    \item input \\
    \begin{equation}
    i_B = f(u_{BE})|_{u_{CE}}
    \end{equation}

    \item output \\
    \begin{equation}
    i_C = f(u_{CE})|_{i_B}
    \end{equation}
        
\end{enumerate}

\subsubsection{MOS}
\begin{enumerate}
    \item characteristics
    \begin{figure}[htb!]
    \centering
    \includegraphics[width=0.6\textwidth]{plots/mosfet_symbol_characteristics}
    \end{figure}

\end{enumerate}

\subsubsection{JFET}
    \begin{enumerate}
        \item Structure \& characteristics
        \begin{figure}[htb!]
            \centering
            \subfigure[]{
                \includegraphics[width=0.4\textwidth, origin=c]{plots/Structure-of-JFET}
            }
            \subfigure[]{
                \includegraphics[width=0.4\textwidth, origin=c]{plots/N-Channel-JFET-Output-and-Transfer-Characteristics-}
            }
        \end{figure}
    \end{enumerate}

\subsubsection{key parameter}
\begin{enumerate}
    \item DC \\
    $V_{gs}(th)$, $V_{gs}(off)$, $I_{dss}$, $R_{gs}(DC)$ 
    \item AC \\
    $g_m = \frac{\partial i_D}{\partial V_{gs}}|_{V_{ds}}$
\end{enumerate}

\subsection{Amplifier}
\begin{enumerate}
    \item Direct coupling
    \item RC coupling
    \item Th$\acute{\text{e}}$venin theorem (Norton Equation)\\
    "Any linear electrical network containing only voltage sources, current sources and resistances 
    can be replaceed at terminals A-B by an equivalent combination of a voltage source $V_{th}$ in 
    a series connection with a resistance $R_{th}$"
    \begin{itemize}
        \item The equivalent voltage $V_{th}$ is the voltage obtained at terminals A-B of the network with
        terminal A-B open circuited.
        \item The equivalent Resistance $R_{th}$ is the resistance that the circiut between terminals A and 
        B would have if all ideal \textbf{voltage sources} in the circuit were replced by a \textbf{short circuit}
        and all ideal \textbf{current sources} were replaced by an \textbf{open circuit}.
        \item If terminals A and B are connected to one another (short), then the current flowing from A and B 
        \item will be $\frac{V_{th}}{R_{th}}$ according to the \textbf{Th$\acute{\text{e}}$venin equivalent circuit}.
        This means that $R_{th}$ could alternatively be calculated as $V_{th}$ devided by the short-curcuit current 
        between A and B when they are connected together. 

    \end{itemize}
\end{enumerate}

\end{document}
