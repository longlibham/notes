% this is a Latex manuscript

%=======================================================================
%   Copyright (C) 2025 Univ. of Bham  All rights reserved.
%   
%   		FileName:		notes_on_particle_physics.tex
%   	 	Author:		LongLI <long.l@cern.ch>
%   		Time:			2025.10.23
%   		Description:
%
%======================================================================

%defination 
\documentclass[a4paper,12pt]{article}


%package
\usepackage{geometry}
\usepackage{graphicx}
\usepackage{fullpage}
\usepackage{siunitx}
\usepackage{amsmath}
\usepackage{verbatim}
\usepackage{overpic}
\usepackage{url}
\usepackage{xspace}
\usepackage{subfigure}
\usepackage{caption}
\usepackage{hyperref}
\usepackage{svg}
\renewcommand{\thefootnote}{\fnsymbol{footnote}}

\graphicspath{{plots/}}

\sisetup{
	free-standing-units=true,
	space-before-unit=true,
	use-xspace=true
}

% hype link setting
\hypersetup{colorlinks,breaklinks,
  linkcolor=black,citecolor=blue,
  filecolor=blue,urlcolor=blue,
  pdfpagemode=UseNone
}

%fot list
\usepackage{listings}
\usepackage{color, xcolor}
\usepackage{listings}
%\lstset{emph={define}, emphstyle=\color{yellow}}
\lstset{backgroundcolor=\color{white}}
\lstset{language=Python, 
    breaklines=true,
    basicstyle=\ttfamily\footnotesize, 
    keywordstyle=\color{keywords},
    commentstyle=\color{comments},
    stringstyle=\color{red},
    showstringspaces=false,
    identifierstyle=\color{blue},
    %procnamekeys={def,class}
    morekeywords={*,...,-},
    numbers=left,
    xleftmargin=2em,
    frame=single,
    framexleftmargin=1.5em
}



% personal macros
\newcommand{\eq}[1]{\eqref{eq:#1}}
\newcommand{\tab}[1]{Table~\ref{tab:#1}}
\newcommand{\fig}[1]{Figure~\ref{fig:#1}}
\newcommand{\sect}[1]{Section~\ref{sect:#1}}


%title
\title{Long's notes and thoughts on Particle Physics}
\author{Long LI\footnote{long.l@cern.ch}}
%\emailAdd{long.l@cern.ch}
\date{Oct. 2025}

\begin{document}


%Add content for page two here (useful for two-sided printing)
\thispagestyle{empty}
\maketitle

\tableofcontents
\newpage

\section{Book: An Introduction To The Standard Model Of Particle Physics}
\label{sect:book1}

\subsection{The construction of the Standard Model}
\begin{enumerate}
    \item Bosons
        \begin{itemize}
            \item Scalar \\
            $J^{P} = 0^{+}$, spin 0, parity +1. Example: Higgs Boson
            \item Pseudo-Scalar \\
            $J^P = 0^-$, example Pions, Kaons, $\eta , \eta^{'}$
            \item Vector \\
            $J^P = 1^{-}$, example: $\gamma$, W, Z, $\rho$, $\Omega$, $J/\Psi$, $\Upsilon$.
        \end{itemize}
    \item Deep In-elastic Scattering (DIS)
        \begin{itemize}
            \item $Q^2$ \\
            Negative square of the four-momentum transferred from the incoming electron to the target proton via the virtual photon.\\
            \begin{equation}
                Q^2 = -q^2
            \end{equation}
            q, four-momentum of the virtual photon ($q = p_{electron-in} - p_{electron-out}$)  
            \item Bjorken Scaling variable \\
            \begin{equation}
                x = \frac{Q^2}{2pq}
            \end{equation}
            Physically represents the fraction of the proton's momentum carried by the parton that the photon hit. $p$, the four-momentum of the target proton.
        \end{itemize}
        
\end{enumerate}

\subsection{Lorentz transformations}
\begin{enumerate}
    \item Postulates
        \begin{itemize}
            \item Principle of relativity: Physical laws are the same in all inertial frames
            \item Constancy of light speed: light propagates at speed c in all inertial frames
        \end{itemize}
        \begin{figure}[htb!]
            \centering
            \includegraphics[width=0.4\textwidth]{lorentz_transformation}
            \caption{$S^{\prime}$ is moving along X axis with V w.r.t S}
            \label{fig:lt:a}
        \end{figure}
    \item Lorentz transformation derivation
    As in \fig{lt:a}, $S^{\prime}$ is moving along X axis with speed v w.r.t S frame. \\
    \textbf{Principle 1 linear relation} between frame S and frame $S^{\prime}$.\\
    \begin{equation}
        \begin{cases}
            y^{\prime} = y, \\
            z^{\prime} = z, \\
            t^{\prime} = At + Bx, \\
            x^{\prime} = Dt + Ex \\
        \end{cases}
    \end{equation}
    A, B, D and E are coefficients.\\
    \textbf{Principle 2 principle of relativity:}\\
    $S^{\prime}$ is moving w.r.t S in volicity V, Inversely, S is moving w.r.t $S^{\prime}$ in volicity -V.
    \begin{equation}
        \begin{cases}
            t = A(-v)t^{\prime} + B(-v) x^{\prime}, \\
            x = D(-v)t^{\prime} + E(-v) x^{\prime} \\
        \end{cases}
    \label{eq:lt:a}
    \end{equation} 
    \begin{itemize}
        \item A(v): relates time in $S^{\prime}$ to time in S, shold out change with volicity. A(v) = A(-v)
        \item B(v): relates space in $S^{\prime}$ to time in S, should change sign with volicity. B(v) = - B(-v) 
        \item D(v): relates time in $S^{\prime}$ to space in S, should change sign with volicity. D(v) = - D(-v)
        \item E(v): relates space in $S^{\prime}$ to space in S, should not change with volicity. E(v) = E(-v)
    \end{itemize}
    \begin{equation}
        \begin{cases}
            t = At^{\prime} - Bx^{\prime}, \\
            x = -Dt^{\prime} + Ex^{\prime} \\
        \end{cases}
        \label{eq:lt:b}
    \end{equation}
    Substitute \eq{lt:b} into \eq{lt:a},
    \begin{equation}
        \begin{cases}
            A^2 - BD = 1, AB - BE = 0, \\
            ED - DA = 0, E^2 -DB = 1 \\
        \end{cases}
    \label{eq:lt:c}
    \end{equation}
    Then,
    \begin{equation}
        B(A - E) = 0, \\
        B = 0 \, \text{or}\, A = E\\
    \end{equation}
    B = 0 means Galilean transformation, and it makes no sense. 
    \begin{equation}
        \begin{cases}
            t = At^{\prime} - Bx^{\prime}, \\
            x = -Dt^{\prime} + Ax^{\prime}, \\
            t^{\prime} = At + Bx, \\
            x^{\prime} = Dt + Ax \\
        \end{cases}
    \end{equation}

    \textbf{principle 3: volicity of light keeps constant in all inertial frame}, x =ct, 
    $x^{\prime}$ = c$t^{\prime}$.

    \begin{equation}
        \begin{cases}
        c(At + Bct) = Dt + Act, \\
        c^2B = D
        \end{cases}
    \end{equation}

    \textbf{Initial states: $x^{\prime} = 0$}
    \begin{equation}
        \begin{aligned}
        0 = Dt + Ax, \\
        v = \frac{x}{t} = -\frac{D}{A},\\
        D = -Av, \\
        B = -\frac{Av}{c^2}
        \end{aligned}
    \end{equation}

    Then, 
    \begin{equation}
        \begin{cases}
        t^{\prime} = A(t - \frac{v}{c^2}x), \\
        x^{\prime} = A(-vt + x)\\
        \end{cases}
    \end{equation}

    And, 
    \begin{equation}
    (ct^{\prime})^2 - (x^{\prime})^2 = (ct)^2 - x^2
    \end{equation}

    So, \\
    \begin{equation}
        \begin{aligned}
            A^2(-vt + x)^2 -c^2A^2(t - \frac{v}{c^2}x)^2 = c^2t^2 - x^2 \\
            A^2[(c^2 - v^2)(\frac{x^2}{c^2} - t^2)] = c^2t^2 - x^2 \\
            \frac{A^2}{c^2}[(c^2 - v^2)(x^2 - c^2t^2)] = c^2t^2 - x^2 \\
            \frac{A^2}{c^2}[(c^2 - v^2) - 1](x^2 - c^2t^2) = 0\\
            \frac{A^2}{c^2}(c^2 - v^2) = 1\\
            A^2 = \frac{c^2}{c^2 - v^2}\\
            A = \frac{1}{\sqrt{1 - \frac{v^2}{c^2}}} = \gamma
        \end{aligned}
    \end{equation}
    Finally,
    \begin{equation}
        \begin{cases}
        t^{\prime} = \gamma(t - \frac{v}{c^2}x), \\
        x^{\prime} = \gamma(-vt + x), \\
        y^{\prime} = y, \\
        z^{\prime} = z
        \end{cases}
    \end{equation}

    \item Einstein summation convention
    \item Proper Lorentz group ($L^{\mu}_{\nu}$)
    \item Metric tensor ($g_{\mu\nu}$) and contravariant metric tensor ($g^{\mu \nu}$)
    \begin{equation}
        g_{\mu\nu} = g^{\mu\nu} = \begin{pmatrix}
            1 & 0 & 0 & 0 \\
            0 & -1 & 0 & 0 \\
            0 & 0 & -1 & 0 \\
            0 & 0 & 0 & -1 
        \end{pmatrix}
    \end{equation}

\end{enumerate}

\subsubsection{Scalars, contravariant and covariant four-vectors}
    \begin{enumerate}
        \item Scalars: invariant under Lorentz transformations \\
        \begin{equation}
            (\Delta s)^2 = (\Delta s^{\prime})^2 = g_{\mu\nu}\Delta x^{\mu}\Delta x^{\nu}
        \end{equation}
        \item contravariant four-vector: $a^{\mu}$ under a proper Lorentz transformation 
        \begin{equation}
            {a^{\prime}}^{\mu} = L^{\mu}_{\nu} a^{\nu}
        \end{equation}
        Usually to describe displacements
        example: energy-momentum vector of a particle (E/c, \textbf{p}).

        \item covariant four-vector \\
        Usually for gradient. \\
        \begin{equation}
            a_{\mu} = g_{\mu\nu}a^{\nu}
        \end{equation}
        

        then, 
        \begin{equation}
            (\Delta s)^2 = g_{\mu\nu}\Delta x^{\mu} \Delta x^{\nu} = \Delta x_{\nu} \Delta x^{\nu}
        \end{equation}

        \item $\alpha^{\mu} = g^{\mu\nu}a_{\nu}$

    \end{enumerate}


\end{document}
